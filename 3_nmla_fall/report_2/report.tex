\documentclass[a4paper,12pt]{article}

\usepackage{amsthm} % в т.ч. для оформления доказательств
\usepackage{cmap}                    % поиск в PDF
\usepackage[T2A]{fontenc}            % кодировка
\usepackage[utf8]{inputenc}            % кодировка исходного текста
\usepackage[english,russian]{babel}    % локализация и переносы
\usepackage{mathtext}                 % русские буквы в фомулах
\usepackage{amsmath,amsfonts,amssymb,amsthm,mathtools} % AMS
\usepackage{icomma} % "Умная" запятая: ,2$ --- число, , 2$ --- перечисление
\usepackage{graphicx} % чтобы вставлять картинки

\usepackage{setspace}
\usepackage{blindtext}

\usepackage{color}
\definecolor{mine}{RGB}{10,54,100}
\definecolor{Violet}{RGB}{120,80,120}
\usepackage[unicode, colorlinks, urlcolor=mine, linkcolor=Violet, pagecolor=Violet]{hyperref}

\usepackage{tikz}
\usetikzlibrary{graphs}
\usetikzlibrary{automata}
\usepackage{pgfplots}
\pgfplotsset{compat=1.9}

%% Номера формул
\mathtoolsset{showonlyrefs=true} % Показывать номера только у тех формул, на которые есть \eqref{} в тексте.

%% Шрифты
\usepackage{euscript}     % Шрифт Евклид
\usepackage{mathrsfs} % Красивый матшрифт

%% Свои команды
\DeclareMathOperator{\sgn}{\mathop{sgn}}
\newcommand\norm[1]{\lVert#1\rVert}
\newcommand\normx[1]{\Vert#1\Vert}

%% Перенос знаков в формулах (по Львовскому)
\newcommand*{\hm}[1]{#1\nobreak\discretionary{}
    {\hbox{$\mathsurround=0pt #1$}}{}}

\newcommand{\RomanNumeralCaps}[1]
{\MakeUppercase{\romannumeral #1}}

\usepackage{listings}
\usepackage{color}

\definecolor{mygray}{rgb}{0.4,0.4,0.4}
\definecolor{mygreen}{rgb}{0,0,10}
\definecolor{myorange}{rgb}{1.0,0.4,0}

\lstdefinestyle{chstyle}{ %
    backgroundcolor=\color{gray!12},
    basicstyle=\ttfamily\small,
    commentstyle=\color{green!60!black},
    keywordstyle=\color{magenta},
    stringstyle=\color{blue!50!red},
    showstringspaces=false,
    %captionpos=b,
    numbers=left,
    numberstyle=\footnotesize\color{gray},
    numbersep=10pt,
    stepnumber=1,
    tabsize=2,
    frame=L,
    framerule=1pt,
    rulecolor=\color{red},
    breaklines=true,
    inputpath=code
}

\usepackage{wasysym}
\usepackage{amssymb}

\usepackage{listings}
\usepackage{color}

\definecolor{mygray}{rgb}{0.4,0.4,0.4}
\definecolor{mygreen}{rgb}{0,0,10}
\definecolor{myorange}{rgb}{1.0,0.4,0}

\lstdefinestyle{chstyle}{ %
    backgroundcolor=\color{gray!12},
    basicstyle=\ttfamily\small,
    commentstyle=\color{green!60!black},
    keywordstyle=\color{magenta},
    stringstyle=\color{blue!50!red},
    showstringspaces=false,
    %captionpos=b,
    numbers=left,
    numberstyle=\footnotesize\color{gray},
    numbersep=10pt,
    stepnumber=1,
    tabsize=2,
    frame=L,
    framerule=1pt,
    rulecolor=\color{red},
    breaklines=true,
    inputpath=code
}


\pagecolor{white}
\author{\href{https://github.com/ypodlesov}{Егор Подлесов}}
\title{Отчет о выполненнном задании по численным методам линейной алгебры.}

\begin{document}

\maketitle
\clearpage

\section*{Краткий обзор.}
Требуется:
\begin{enumerate}
    \item Реализовать метод для оценки спектра матрицы и итерационный метод Чебышева. 
    \item Оценить спектр матрицы.
    \item Определить количество итераций в итерационном методе Чебышева так, чтобы погрешность была меньше чем у прямого метода вращений Гивенса.
    \item Среднеквадратическую норму погрешности решения прямого метода и метода Чебышева, относительную погрешность решения метода Чебышева.
    \item График среднеквадратической нормы погрешности решения.
\end{enumerate}

\section*{Реализация.}
\begin{lstlisting}[style=chstyle, language=C, numbers=none]
template <typename T>
TVector<T> GetParameters(const TMatrix<T>& a, size_t iter_num) {
    TVector<T> result(iter_num);
    result(0) = 1;
    for (size_t i = 1; i < iter_num; i *= 2) {
        TVector<T> tmp(iter_num);
        for (size_t j = 0; j < 2 * i; ++j) {
            if (j & 1) {
                tmp(j) = (i << 2) - result((j - 1) >> 1);
            } else {
                tmp(j) = result(j >> 1);
            }
        }
        result = std::move(tmp);
    }
    const auto [spectrum_min, spectrum_max] = a.GetSpectrumBoundaries();
    const auto r_0 = (spectrum_max - spectrum_min) / (spectrum_max + spectrum_min);
    const auto tau_0 = 2 / (spectrum_min + spectrum_max);
    for (size_t i = 0; i < iter_num; ++i) {
        result(i) = -std::cos(result(i) / (iter_num << 1) * M_PI);
        result(i) = tau_0 / (1 + result(i) * r_0);
    }
    return result;
}

template <typename T>
bool SolveSystem(const TMatrix<T>& a, const TVector<T>& b, TVector<T>& result, const size_t iter_num) {
    if (a.GetSize1() != a.GetSize2()) {
        return false;
    }
    auto methodParams = GetParameters(a, iter_num);
    result = TVector<T>(b.GetSize());
    result.Nullify();
    for (size_t iter = 0; iter < iter_num; ++iter) {
        TVector<T> tmp(result.GetSize());
        tmp.Nullify();
        for (size_t i = 0; i < a.GetSize1(); ++i) {
            double v = 0;
            for (size_t j = 0; j < a.GetSize2(); ++j) {
                v += a(i, j) * result(j);
            }
            tmp(i) = (b(i) - v) * methodParams(iter_num) + result(i);
        }
        result = std::move(tmp);
    }
    return true;
}

template <typename T>
std::pair<T, T> TMatrix<T>::GetSpectrumBoundaries() const {
    double left = 0;
    double right = 0;
    for (size_t i = 0; i < Size1; ++i) {
        double r = 0;
        for (size_t j = 0; j < Size2; ++j) {
            if (j == i) {
                continue;
            }
            r += fabs(Data[i * Size2 + j]);
        }
        left = std::min<T>(left, Data[i * Size2 + i] - r);
        right = std::max<T>(right, Data[i * Size2 + i] + r);
    }
    return std::make_pair(left, right);
}
\end{lstlisting}

\section*{Оценка спектра матрицы.}
Спектр матрицы будем оценивать по теореме Гершгорина.

Положим  
$$R_i = \sum_{i \ne j} |a_{ij}|$$
$$D(a_{ii}, R_i) \text{ - круг с центром в точке } a_{ii} \text{(для комплексного случая) и радиусом} R_i$$

\textbf{Теорема Гершгорина}

Каждое собственное значение матрицы лежит хотя бы в одном круге Гершгорина.

\section*{Сравнение прямого метода вращений и итерационного метода Чебышева.}

\begin{center}
    \begin{tabular}{ c c c }
     Метод решения & Среднеквадратичная погрешность & Относительная погрешность \\ 
     Метод Гаусса & 0.00359501 & 0.00792001 \\
     Метод Чебышева, 1 & 0.010141 & 0.0311231 \\
     Метод Чебышева, 2 & 0.267352 & 0.801734 \\
     Метод Чебышева, 4 & 0.128336 & 0.398931 \\
     Метод Чебышева, 8 & 0.04401372 & 0.129781 \\
     Метод Чебышева, 16 & 0.00301129 & 0.00917512 \\
     Метод Чебышева, 32 & 3.19325e-05 & 5.67183e-05 \\
     Метод Чебышева, 64 & 7.74915e-10 & 2.974971e-09 \\
     Метод Чебышева, 128 & 9.01374e-19 & 3.83482e-18 \\
    \end{tabular}
\end{center}

\begin{tikzpicture}
    \begin{axis}[
        title = Exponenta,
        xlabel = {Количество итераций в методе Чебышева},
        ylabel = {Среднеквадратичная погрешность},
        minor tick num = 1
    ]
        \addplot coordinates {
            (1, 0.010141) (2, 0.267352) (4, 0.128336) (8, 0.04401372) (16, 3.19325e-05) (32, 3.19325e-05) (64, 7.74915e-10) (128, 9.01374e-19)
        };
    \end{axis}
\end{tikzpicture}

Из вышеописанного видно, что итерационный метод выигрывает у прямого метода решения СЛАУ (метода вращений) при количестве итераций $iter\_num \ge 32$

\section*{Вывод.}
Реализован метод итераций Чебышева. Оценена его сходимость. Проведено сравнение с прямым методом вращений.  




\end{document}