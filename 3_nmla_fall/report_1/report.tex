\documentclass[a4paper,12pt]{article}

\usepackage{amsthm} % в т.ч. для оформления доказательств
\usepackage{cmap}                    % поиск в PDF
\usepackage[T2A]{fontenc}            % кодировка
\usepackage[utf8]{inputenc}            % кодировка исходного текста
\usepackage[english,russian]{babel}    % локализация и переносы
\usepackage{mathtext}                 % русские буквы в фомулах
\usepackage{amsmath,amsfonts,amssymb,amsthm,mathtools} % AMS
\usepackage{icomma} % "Умная" запятая: ,2$ --- число, , 2$ --- перечисление
\usepackage{graphicx} % чтобы вставлять картинки

\usepackage{setspace}
\usepackage{blindtext}

\usepackage{color}
\definecolor{mine}{RGB}{10,54,100}
\definecolor{Violet}{RGB}{120,80,120}
\usepackage[unicode, colorlinks, urlcolor=mine, linkcolor=Violet, pagecolor=Violet]{hyperref}

\usepackage{tikz}
\usetikzlibrary{graphs}
\usetikzlibrary{automata}

%% Номера формул
\mathtoolsset{showonlyrefs=true} % Показывать номера только у тех формул, на которые есть \eqref{} в тексте.

%% Шрифты
\usepackage{euscript}     % Шрифт Евклид
\usepackage{mathrsfs} % Красивый матшрифт

%% Свои команды
\DeclareMathOperator{\sgn}{\mathop{sgn}}
\newcommand\norm[1]{\lVert#1\rVert}
\newcommand\normx[1]{\Vert#1\Vert}

%% Перенос знаков в формулах (по Львовскому)
\newcommand*{\hm}[1]{#1\nobreak\discretionary{}
    {\hbox{$\mathsurround=0pt #1$}}{}}

\newcommand{\RomanNumeralCaps}[1]
{\MakeUppercase{\romannumeral #1}}

\usepackage{listings}
\usepackage{color}

\definecolor{mygray}{rgb}{0.4,0.4,0.4}
\definecolor{mygreen}{rgb}{0,0,10}
\definecolor{myorange}{rgb}{1.0,0.4,0}

\lstdefinestyle{chstyle}{ %
    backgroundcolor=\color{gray!12},
    basicstyle=\ttfamily\small,
    commentstyle=\color{green!60!black},
    keywordstyle=\color{magenta},
    stringstyle=\color{blue!50!red},
    showstringspaces=false,
    %captionpos=b,
    numbers=left,
    numberstyle=\footnotesize\color{gray},
    numbersep=10pt,
    stepnumber=1,
    tabsize=2,
    frame=L,
    framerule=1pt,
    rulecolor=\color{red},
    breaklines=true,
    inputpath=code
}

\usepackage{wasysym}
\usepackage{amssymb}
\pagecolor{white}
\author{\href{https://github.com/ypodlesov}{Егор Подлесов}}
\title{Отчет о выполненнном задании по численным методам линейной алгебры.}

\begin{document}

\maketitle
\clearpage

\section*{Краткий обзор.}

Сатья показывает результаты реализации и замеров времени работы алгоритма решения \textbf{системы линейных алгебраических уравнений} методом \textbf{вращений Гивенса}.

\begin{lstlisting}[caption=Схема проекта, style=chstyle, language=bash]
cmc-numerical-methods-of-linear-algebra
|-- benchmark
|   --- CMakeLists.txt
|   --- givens_rotations.cpp
|-- CMakeLists.txt
|-- src
|   |-- CMakeLists.txt
|   |-- common.h
|   |-- givens_rotations.cpp
|   |-- givens_rotations.h
|   |-- matrix.cpp
|   |-- matrix.h
|   |-- triangular_matrix.cpp
|   |-- triangular_matrix.h
|   |-- vector.cpp
|   |-- vector.h
|-- tests
    |-- CMakeLists.txt
    |-- givens_rotations_ut.cpp
    |-- SLAU_var_5.csv
\end{lstlisting}

\begin{itemize}
    \item \textbf{src} - реализация классов \textbf{TMatrix}, \textbf{TVector}, а также решения \textbf{СЛАУ} вращениями Гивенса.  
    \item \textbf{benchmark} - замер времени выполнения алгоритма.
    \item \textbf{tests} - тестирование алгоритма.
\end{itemize}

\clearpage
\section*{Постановка задачи.}
\textbf{Дано:} Невырожденная матрица $A$ размеров 100 $\times$ 100. \\
\textbf{Требуется:} 

    \begin{enumerate}
        \item Реализовать алгоритм решения \textbf{СЛАУ} с коэффициентами - элементами данной матрицы, используя метод вращений Гивенса. Вектор-результат $b$ сгенерировать из случайных чисел на отрезке $[-1, 1]$ с равномерным распределением. 
То есть надо решить систему вида 
$$ Ax = b \text{,} $$
где $ A \in \mathbb{R}^{100 \times 100}\text{, }b \in \mathbb{R}^{100} $. 
        \item Протестировать программу - убедиться что найденный вектор $x$ удовлетворяет следующему неравенству $ \norm{Ax - b} < \varepsilon $, для некоторого достаточно малого $ \varepsilon $. 
        \item  Замерить время работы программы на входной матрице.
    \end{enumerate}

\section*{Метод вращений Гивенса.}
Метод заключается в применении последовательных операций умножения на матрицы специального вида - так называемые матрицы вращений. Вследствии чего исходная матрица получает треугольный вид. 

Матрица вращения имеет следующий вид: 
\[
  Q_{ij} =
  \left[ 
    {
        \begin{array}{ccc}
            q_{ii} & \ldots & q_{ij} \\ 
            \vdots & \ddots & \vdots \\
            q_{ji} & \ldots & q_{jj}
        \end{array} 
    } 
  \right]
\]
То есть у матрицы $ Q_{ij} $ на всех позициях кроме 
    $ (i, i), \; (i, j), \; (j, i), \; (j, j) $ стоят нулевые элементы. 
При этом ненулевые элементы подчиняются слеющим соотношениям: 
    $$
        \begin{cases}
            q_{ii} = q_{jj} \\ 
            q_{ji} = -q_{ij} \\
            q_{ii}^2 + q_{ij}^2 = 1
        \end{cases}
    $$

\textbf{Замечание:} Можно положить, $ q_{ii} = cos(\alpha) $ и $ q_{ij} = sin(\alpha) $. Поэтому матрица $ Q_{ij} $ называется матрицей поворота на угол $ \alpha $.

Таким образом можно занулять поддиагональные элементы в каждой колонке. 
Для $ j $-й колонки зануление поддиагонального элемента в строке $ i $ подразумевает матрицу поворота $ Q_{ji} $ с следующими элементами:
        
$$
\begin{cases}
    q_{jj} = \frac{a_{jj}}{\sqrt{a_{jj} + a_{ij}}} \\
    q_{ji} = \frac{a_{ij}}{\sqrt{a_{jj} + a_{ij}}}
\end{cases}
$$

Проверив подстановкой, убеждаемся что элемент $ a_{ij} $ зануляется.

Сделав это для всех поддиагональных элементов мы получаем разложение: 

$$
A = QR
$$
где $ Q $ - унитарная (ортогональная) матрица, $ R $ - верхняя треугольная матрица.

Система 
$$
Ax = b 
$$
полчает вид 
$$
Rx = Q^Tb
$$
Такая система решается просто. Последовательно находим $ x_n,\; x_{n-1},\; \ldots,\; x_1 $. 


\section*{Краткое описание реализации.}

    Основные функции
    \begin{itemize}
        \item \textbf{NTriangularMatrix::SolveSystem} - решение СЛАУ с треугольной матрицей.
        \item \textbf{NGivensRotations::SolveSystem} - решение СЛАУ вращениями Гивенса.
        \item \textbf{NGivensRotations::SystemToTriangular} - приведение матрицы к треугольной форме вращениями Гивенса.
    \end{itemize}
    
    Вся реализация сохранена в репозитории \textbf{GitHub: \href{https://github.com/ypodlesov}{ypodlesov}}

\section*{Тест и замер времени работы программы.}

    Тест реализован в файле \textbf{tests/givens\_rotations\_ut.cpp}. Он проверяет, что вектор решение $ x $ при умножении на матрицу $ A $ отличается от вектора $ b $ по $l_2$ норме меньше чем на $ \varepsilon = 10^{-6}$. 
    
    На данной в входном файле матрице $ A \in \mathbb{R}^{100 \times 100} $ программа показывает следующие результаты: 

    \begin{lstlisting}[caption=Результаты работы программы, style=chstyle, language=bash]
Run on (32 X 1995.31 MHz CPU s)
CPU Caches:
L1 Data 32 KiB (x32)
L1 Instruction 32 KiB (x32)
L2 Unified 4096 KiB (x16)
L3 Unified 16384 KiB (x1)
Load Average: 0.28, 0.18, 0.11
---------------------------------------------------------
Benchmark               Time             CPU   Iterations
---------------------------------------------------------
BM_SolveSystem   67812391 ns     67810825 ns           16
    \end{lstlisting}

    То есть $ \approx 0.067 $ секунды.

\end{document}